
The use of increasing levels of parallelism and concurrency in HPC and Cloud
is a double-edged sword: while it helps increase performance, save energy, and
enhance user experience, it
unfortunately comes with the cost of concurrency bugs.
%
This trend can worsen with the upcoming HPC/Cloud
integration~\cite{dan-herbein-dong} unless we accelerate the
development of concurrency debugging methods.
%
However, unless these debugging systems themselves are
simple and widely applicable,
the maintenance of multiple debugging approaches
will also prove to be a burden.
%
Driven by these observations,
in this paper we take a feature-rich language, namely Go~\cite{go-citation},
research how to develop an effective debugging methods for real-world
Go programs, and present the resulting new tool called GoAT.
%
The main contribution of this work is to demonstrate how
GoAT's static and dynamic analysis methods---while very well known
by their names and effectiveness in simpler contexts---can
be brought to bear on realistic bug-scenarios in the setting of
a complex language such as Go; this is our central contribution.


Go enjoys accelerating acceptance in a wide variety of
communities---especially the Cloud community.
%
It involves shared memory, message passing, nondeterministic
message reception and selection,
dynamic process creation,
and programming styles that tend to create thousands of Go routines
and discard them to be garbage collected when not needed.
%
This combination of features is known to be the reason why Go
is popular; yet, it is also well recognized that these features also make
Go debugging hard.
%
Our work is especially relevant considering that
there are there no widely usable tools for debugging Go; even well-curated
bug benchmark suites are only just
now beginning to appear~\cite{tu-concurrentBugs-asplos19,yuan-gobench-cgo21}.
%
Combinations of Go's features are found in many other languages,
making our study carry a broader degree of relevance
to the HPC/Cloud ecosystem, in addition to providing an effective
tool for Go itself.


One of our priorities in designing GoAT
was to reuse as much of the existing Go infrastructure
as we could.
%
The static analysis component relies on source rewriting, inserting
{\em potential} schedule yield points
at all
``interesting'' (concurrency-relevant) spots inside a given Go program.
%
The dynamic component consists of yielding control back
to the Go runtime with a user-settable probability value.
%
Last but not least, we also exploited the tracing features of Go
to provide, via GoAT, (1)~as much debug information as possible, and
(2)~use the tracing features to measure and report concurrency coverage.


Our preliminary results in terms of this combined static/dynamic
approach yielded encouraging results.
%
However, bug-hunting on curated bug data sets cannot, by itself, give any
indications on how well an approach might {\em generalize}.
%
Generalization in terms of debugging tools is often achieved
by carefully defining {\em coverage metrics} and showing how
well these metrics are met.
%
Again, we walked into a field with very few such coverage metrics.


Our coverage metrics
are clearly inspired by other concurrrency coverage
metrics~\cite{cite,a,few,of,them}.
%
Overall, our coverage metrics are inspired by real-world Go bugs,
and this is new in terms of debugging Go concurrrency.
%
Also coverage metrics for select statements---high in the list of
Go bugs---are believed to be novel.
%
Last but not least, we also differentiate between covering
easier bugs and rare bugs: the former do not require any
schedule control, while the latter is triggered only through schedule
control.


Developing these coverage metrics required studying the output
of our initial
tracing infrastructure on Go bug benchmarks.
%
We take advantage of our own tracing framework to
tabulate and report on the achieved coverage metrics.
%
All this is offered via a new tool called \goat (Go Analysis and Testing Tool)
that we now describe, after introducing relevant features of Go
itself.


\noindent{\bf Features of Go:\/}
Go \cite{go} is a statically typed language initially developed by Google and at present widely used by many.
%
It employs channel-based Hoare's Communicating Sequential Processes (CSP) \cite{hoare-csp78} semantics in its core and provides a productivity-enhancing environment for concurrent programming.
%
The concurrent model in Go centers around
1) \textit{goroutines} as light-weight user-level threads (processing units),
2) \textit{channels} for explicit messaging to synchronize and share memory through communication, and
3) a \textit{scheduler} that orchestrates goroutine interactions while shielding
the user from
many low-level
aspects of the runtime.
%
This design
facilitates the
construction
of data flow models that efficiently utilize multiple CPU cores.
%
Because of the simple yet powerful concurrency model, many real production software systems take advantage of Go,
including
container software systems such as Docker \cite{merkel2014docker}, Kubernetes \cite{kubernetes},  key-value databases \cite{etcd}, and web server libraries \cite{grpc}.
%


In traditional shared-memory concurrent languages such as Java/C/C++, threads interact with each other via shared memory.
%
Processes in CSP-based languages such as Erlang communicate through mailbox (asynchronous) message passing.
%
Go brings all these features together into one language and encourages developers to \textit{share memory through communication} for safe and straightforward concurrency and parallelism.
%
The visibility guarantee of memory writes is specified in the memory model\cite{go-memModel} under synchronization constraints (\textit{happens-before} partial order \cite{lamport-hb-1978}).
%
The language is equipped with a rich vocabulary of \textit{serialization} features to facilitate the memory model constraints; they include synchronous and asynchronous communication (either unbuffered or buffered channels), memory protection, and barriers for efficient synchronization.
%
This rich mixture of features has, unfortunately, greatly exacerbated the complexity of Go debugging.
%
In fact, the popularity of Go has outpaced its debugging support~\cite{go-survey,tu-concurrentBugs-asplos19,dilley-empirical-saner19}.
%
There are some encouraging developments in support of debugging, such as a data race checker \cite{go-race-blog} that has now become a standard feature of Go, and has helped catch many a bug.
%
However, the support for ``traditional concurrency debugging'' such as detecting atomicity violations and Go-specific bug-hunting support for Go idioms (e.g., misuse of channels and locks) remain insufficiently addressed.
%

\noindent{\bf Contributions:\/} In
this work, we present the initial steps that we took towards addressing this lack.
%
Since a bug might
occur
at various levels of abstraction, dynamic tracing provides a practical and uniform way to track multiple facets of the program during execution (as we have shown in our prior work~\cite{difftrace}).
%
Also, unlike assertion-based tools \cite{lange-staticType-icse18,wolf-gobra-cav21}, a dynamic tool is more automated, not requiring user expertise.
%
We developed a facility that automatically gathers \textit{execution concurrency traces} (\ie sequences of events) during the execution of Go applications with minimal instrumentation.
%
By enhancing the Go built-in tracing mechanism with \textit{concurrency usage} events, we enrich original \textit{execution traces} so that they accurately reflect the dynamic concurrency behavior of applications.
%
Upon Go programs' termination when tracing is enabled, traces are flushed and structurally stored in relational tables of an SQL database, enabling multi-aspect program analysis in offline.
%

With the help of this novel \textit{automated dynamic tracing} mechanism,
we have implemented a testing framework that
\textit{accelerates} bug exposure by manipulating the native scheduler around \textit{critical points} in the code---combination of constructs that heighten the propensity for bug-introduction.
%



To summarize, here are our contributions:
\stcmtside{need to update}
\begin{itemize}
    \item We take the tracing mechanism embedded in the standard Go that captures \textit{execution trace} (ET) and enhance it with a set of concurrency primitive usage events to obtain \textit{execution concurrency trace} (ECT). While the primary usage of ET is performance analysis, ECT provides an accurate and comprehensive model of concurrent execution, enabling automated analysis of logical behavior and concurrent bug detection.
    \item We introduce a framework that automatically instruments the target program, collects ECT, and structurally stores them in a database. Through querying the database, several visualizations and reports are accessible.
    \item We propose an approach to identify the points (\ie source line number) in the target program in which a random noise might drastically change the program's dynamic behavior. We analyze ECTs to identify such points and measure schedule space coverage per execution.
    \item Coverage metric
\end{itemize}
\stcmtside{need to update}
The rest of this paper is as follows: Section \ref{sec:correctness} discusses correctness problems and approaches in Go. Section \ref{sec:ect} describes the enhancement we made to the tracer package. Section \ref{sec:disc} discusses the advantages of dynamic tracing for concurrent debugging and draws the ongoing and future direction of the current work. At last, Section \ref{sec:summary} summarizes and concludes.
