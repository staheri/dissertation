We presented \goat, an analysis and testing framework for concurrent Go applications to assist concurrency debugging of real-world applications.
%
\goat combines static and dynamic methods to model and explore application execution.
%
The scheduler behavior is pertubed with automatically injected random delays to accelerate the exposure of bug, if any.
%
By dynamic measurment of a set of coverage requirements, we quantify the quality of schedule-space exploration of \goat.
%
\goat detects all 68 blocking bugs of GoKer benchmark which are the bug kernels of top nine open-soruce projects written in Go.
%
The schedule perturbation showed effectiveness in accelerating the bug exposure.
%
Proposed coverage requirements accurately reflect the dynamic behavior of program executions and testing iterations.

Engineering of \goat is flexible and extensible to more advanced components.
%
For example, current minimal \goat engine can be extended to take the full control over the Go scheduler and ``guide'' testing towards untested interleaving.
%
We are dockerizing \goat for easy and public use.
%
We want to test on real-world programs.
%
The data that ECT includes is rich enough for training accurate models and apply machine learning methods to learn and predict bug patterns.
