%%% -*-LaTeX-*-
%%% This is the abstract for the thesis.
%%% It is included in the top-level LaTeX file with
%%%
%%%    \preface    {abstract} {Abstract}
%%%
%%% The first argument is the basename of this file, and the
%%% second is the title for this page, which is thus not
%%% included here.
%%%
%%% The text of this file should be about 350 words or less.

With the high growth in computation power and the invention of modern languages, concurrent software testing and debugging is vital to deliver reliable software.
%
Finding bugs in concurrent/parallel software is notoriously challenging because 1) the interleaving space grows exponentially with the number of processing units (\eg, CPU cores), 2) the nondeterministic nature of concurrent software makes concurrent bugs difficult to reproduce, and 3) root-causing misbehaved executions of a concurrent program is nontrivial due to the complex interactions between concurrent components of the program.
%
In this work, we have designed and implemented several frameworks and toolchains to overcome large-scale and real-world concurrent/parallel software debugging challenges.
%
Our methods aim to facilitate the concurrent debugging process by providing efficient data collection and effective information retrieval mechanisms to target real-world software and bugs.
%
First, we introduce \parlot, a whole-program call tracing framework for HPC applications (MPI+X) that highly compresses the traces (up to more than 21000 times) while adding minimal overhead with an average required bandwidth of just 56 kB/s per core.
%
Second, we present DiffTrace, a series of automated data abstraction, representation, and visualization techniques that differentiate the collected \parlot traces and narrows the search space down to just a few candidates of buggy traces.
%
Finally, we illustrate \goat, an end-to-end framework for automated tracing, analysis, and testing of concurrent Go applications.
%
We also propose a set of coverage metrics to measure the quality of schedule exploration in CSP-like concurrent languages.
%
Our evaluation of \goat on a recent Go concurrency bug benchmark of real-world programs shows a 100\% success rate in detecting blocking bugs.
