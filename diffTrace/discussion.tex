DiffTrace is the first tool we know of that situates debugging around {\em whole program}
diffing, and~(1)~provides user-selectable front-end filters of function calls to keep;
~(2)~summarizes loops based on state-of-the-art algorithms to detect loop-level
behavioral differences;
~(3)~condenses the loop-summarized
traces into concept lattices that are built using incremental
algorithms;~(4)~and clusters behaviors using hierarchical clustering and ranks them by similarity to detect and highlight the most salient differences.
%a
We deliberately chose the path of a clean start that addresses missing features
in existing tools and missing collectivism in the debugging community.
%
Our initial assessment of this design is encouraging.
%

In our future work we will improve DiffTrace components as follows:
%
%
(1)~Optimizing them to exploit multicore CPUs, thus reducing the overall analysis time;
%
(2)~Converting ParLOT traces into Open Trace Format (OTF2) by logically timestamping trace entries to mine temporal properties of functions such as \textit{happened-before}~\cite{lamport};
%
(3)~Conducting systematic bug-injection to see whether concept lattices and loop structures can be used as elevated features for precise bug classifications via machine learning and neural network techniques; and
%
(4)~Taking up more challenging and real-world examples to evaluate DiffTrace against similar tools, and release it to the community.
%--end
% leave a blank line

\subsection{Acknowledgements} This work is supported in part by
NSF awards CCF 1817073 and 1704715.
